\batchmode
\makeatletter
\def\input@path{{/Users/stephaniekim/ps5//}}
\makeatother
\documentclass[english]{article}\usepackage[]{graphicx}\usepackage[]{color}
%% maxwidth is the original width if it is less than linewidth
%% otherwise use linewidth (to make sure the graphics do not exceed the margin)
\makeatletter
\def\maxwidth{ %
  \ifdim\Gin@nat@width>\linewidth
    \linewidth
  \else
    \Gin@nat@width
  \fi
}
\makeatother

\definecolor{fgcolor}{rgb}{0.345, 0.345, 0.345}
\newcommand{\hlnum}[1]{\textcolor[rgb]{0.686,0.059,0.569}{#1}}%
\newcommand{\hlstr}[1]{\textcolor[rgb]{0.192,0.494,0.8}{#1}}%
\newcommand{\hlcom}[1]{\textcolor[rgb]{0.678,0.584,0.686}{\textit{#1}}}%
\newcommand{\hlopt}[1]{\textcolor[rgb]{0,0,0}{#1}}%
\newcommand{\hlstd}[1]{\textcolor[rgb]{0.345,0.345,0.345}{#1}}%
\newcommand{\hlkwa}[1]{\textcolor[rgb]{0.161,0.373,0.58}{\textbf{#1}}}%
\newcommand{\hlkwb}[1]{\textcolor[rgb]{0.69,0.353,0.396}{#1}}%
\newcommand{\hlkwc}[1]{\textcolor[rgb]{0.333,0.667,0.333}{#1}}%
\newcommand{\hlkwd}[1]{\textcolor[rgb]{0.737,0.353,0.396}{\textbf{#1}}}%

\usepackage{framed}
\makeatletter
\newenvironment{kframe}{%
 \def\at@end@of@kframe{}%
 \ifinner\ifhmode%
  \def\at@end@of@kframe{\end{minipage}}%
  \begin{minipage}{\columnwidth}%
 \fi\fi%
 \def\FrameCommand##1{\hskip\@totalleftmargin \hskip-\fboxsep
 \colorbox{shadecolor}{##1}\hskip-\fboxsep
     % There is no \\@totalrightmargin, so:
     \hskip-\linewidth \hskip-\@totalleftmargin \hskip\columnwidth}%
 \MakeFramed {\advance\hsize-\width
   \@totalleftmargin\z@ \linewidth\hsize
   \@setminipage}}%
 {\par\unskip\endMakeFramed%
 \at@end@of@kframe}
\makeatother

\definecolor{shadecolor}{rgb}{.97, .97, .97}
\definecolor{messagecolor}{rgb}{0, 0, 0}
\definecolor{warningcolor}{rgb}{1, 0, 1}
\definecolor{errorcolor}{rgb}{1, 0, 0}
\newenvironment{knitrout}{}{} % an empty environment to be redefined in TeX

\usepackage{alltt}
\usepackage[T1]{fontenc}
\usepackage[latin9]{inputenc}
\usepackage{geometry}
\geometry{verbose,tmargin=2.54cm,bmargin=2.54cm,lmargin=2.54cm,rmargin=2.54cm}
\usepackage{babel}
\IfFileExists{upquote.sty}{\usepackage{upquote}}{}
\begin{document}

\title{Stat 243}


\title{Problem Set 5}


\author{Stephanie Kim}


\author{github: stephaniesookim}

\maketitle
/raggedright


\section*{Number 1}

\begin{knitrout}
\definecolor{shadecolor}{rgb}{0.969, 0.969, 0.969}\color{fgcolor}\begin{kframe}
\begin{alltt}
\hlkwd{load}\hlstd{(}\hlstr{"~/stat243-fall-2014/ps/ps5prob1.Rda"}\hlstd{)}

\hlstd{log} \hlkwb{<-} \hlstd{X}\hlopt\hlstd{beta}
\hlstd{p} \hlkwb{<-} \hlkwd{exp}\hlstd{(log)}\hlopt{/}\hlstd{(}\hlnum{1}\hlopt{+}\hlkwd{exp}\hlstd{(log))}
\hlstd{likelihood}\hlkwb{<-}\hlkwd{prod}\hlstd{(}\hlkwd{dbinom}\hlstd{(y,n,p))}
\hlstd{likelihood}
\end{alltt}
\begin{verbatim}
## [1] 0
\end{verbatim}
\begin{alltt}
\hlstd{beta} \hlkwb{<-} \hlkwd{c}\hlstd{(}\hlnum{1}\hlstd{,}\hlnum{2}\hlstd{,}\hlnum{3}\hlstd{,}\hlnum{4}\hlstd{)}
\hlstd{log} \hlkwb{<-} \hlstd{X}\hlopt\hlstd{beta}
\hlstd{p} \hlkwb{<-} \hlkwd{exp}\hlstd{(log)}\hlopt{/}\hlstd{(}\hlnum{1}\hlopt{+}\hlkwd{exp}\hlstd{(log))}
\hlstd{likelihood}\hlkwb{<-}\hlkwd{prod}\hlstd{(}\hlkwd{dbinom}\hlstd{(y,n,p))}
\hlstd{likelihood}
\end{alltt}
\begin{verbatim}
## [1] 0
\end{verbatim}
\end{kframe}
\end{knitrout}

From the above, we can observe that the likelihood is 0 regardless
of the value of beta. 

This is because of the underflow. Since the number is less than 1e-308,
it returns 0.

\begin{knitrout}
\definecolor{shadecolor}{rgb}{0.969, 0.969, 0.969}\color{fgcolor}\begin{kframe}
\begin{alltt}
\hlstd{logsum} \hlkwb{<-} \hlkwd{sum}\hlstd{(}\hlkwd{dbinom}\hlstd{(y,n,p,}\hlkwc{log}\hlstd{=T))}
\hlstd{logsum}
\end{alltt}
\begin{verbatim}
## [1] -19895
\end{verbatim}
\end{kframe}
\end{knitrout}

To avoid this, we can take a log and then sum them up. The result
is -1862.331.


\section*{Number 2 (a)}

\begin{knitrout}
\definecolor{shadecolor}{rgb}{0.969, 0.969, 0.969}\color{fgcolor}\begin{kframe}
\begin{alltt}
\hlkwd{options}\hlstd{(}\hlkwc{digits}\hlstd{=}\hlnum{22}\hlstd{)}
\hlnum{1.000000000001}
\end{alltt}
\begin{verbatim}
## [1] 1.000000000001000088901
\end{verbatim}
\end{kframe}
\end{knitrout}

It prints 1.000000000001000088901. 

Thus we have 16 decimal places of accuracy here.


\section*{Number 2 (b)}

\begin{knitrout}
\definecolor{shadecolor}{rgb}{0.969, 0.969, 0.969}\color{fgcolor}\begin{kframe}
\begin{alltt}
\hlkwd{options}\hlstd{(}\hlkwc{digits}\hlstd{=}\hlnum{22}\hlstd{)}
\hlstd{x} \hlkwb{<-} \hlkwd{c}\hlstd{(}\hlnum{1}\hlstd{,} \hlkwd{rep}\hlstd{(}\hlnum{1e-16}\hlstd{,} \hlnum{10000}\hlstd{))}
\hlkwd{sum}\hlstd{(x)}
\end{alltt}
\begin{verbatim}
## [1] 1.000000000000999644811
\end{verbatim}
\end{kframe}
\end{knitrout}

It prints 1.000000000000999644811.

The sum() does not give the right answer up to the accuracy expected
from part (a).

We have 12 decimal places of accuracy.


\section*{Number 2 (c) : Python}

\begin{knitrout}
\definecolor{shadecolor}{rgb}{0.969, 0.969, 0.969}\color{fgcolor}\begin{kframe}
\begin{alltt}
import numpy as np 
from decimal import Decimal
vec=\hlkwd{np.array}([1e-16]*(10001)) 
vec[0]=1 
\hlkwd{print} (\hlkwd{np.sum}(vec))

\hlcom{## 1.0}
\end{alltt}
\end{kframe}
\end{knitrout}

It prints 1.0.

Summation in Python is even less accurate than in R.


\section*{Number 2 (d)}

\begin{knitrout}
\definecolor{shadecolor}{rgb}{0.969, 0.969, 0.969}\color{fgcolor}\begin{kframe}
\begin{alltt}
\hlkwd{options}\hlstd{(}\hlkwc{digits}\hlstd{=}\hlnum{22}\hlstd{)}
\hlstd{x} \hlkwb{<-} \hlkwd{c}\hlstd{(}\hlnum{1}\hlstd{,} \hlkwd{rep}\hlstd{(}\hlnum{1e-16}\hlstd{,} \hlnum{10000}\hlstd{))}

\hlcom{# first for loop}
\hlstd{fl1} \hlkwb{<-} \hlnum{0}
\hlkwa{for} \hlstd{(i} \hlkwa{in} \hlnum{1}\hlopt{:}\hlkwd{length}\hlstd{(x)) \{}
  \hlstd{fl1} \hlkwb{<-} \hlstd{fl1} \hlopt{+} \hlstd{x[i]}
\hlstd{\}}
\hlstd{fl1}
\end{alltt}
\begin{verbatim}
## [1] 1
\end{verbatim}
\begin{alltt}
\hlcom{# second for loop: do the summation with 1 as the last value in the vector}
\hlstd{fl2} \hlkwb{<-} \hlnum{0}
\hlkwa{for} \hlstd{(i} \hlkwa{in} \hlnum{2}\hlopt{:}\hlkwd{length}\hlstd{(x)) \{}
  \hlstd{fl2} \hlkwb{<-} \hlstd{fl2} \hlopt{+} \hlstd{x[i]}
\hlstd{\}}
\hlstd{fl2} \hlkwb{<-} \hlstd{fl2} \hlopt{+} \hlstd{x[}\hlnum{1}\hlstd{]}
\hlstd{fl2}
\end{alltt}
\begin{verbatim}
## [1] 1.000000000001000088901
\end{verbatim}
\end{kframe}
\end{knitrout}

The first for loop returns a wrong value (1).

On the other hand, the second for loop returns a value that has 16
decimal places of accuracy.


\section*{Number 2 (d) : Python}

\begin{knitrout}
\definecolor{shadecolor}{rgb}{0.969, 0.969, 0.969}\color{fgcolor}\begin{kframe}
\begin{alltt}
vec = \hlkwd{np.array}([1e-16]*(10001)) 
vec[0] = 1.0

\hlcom{#first for loop}
fl1 = 0 
for i in \hlkwd{range}(\hlkwd{len}(vec)):
  fl1 += vec[i] 
\hlkwd{print}(\hlkwd{Decimal}(fl1))

\hlcom{# second for loop: do the summation with 1 as the last value value in the vector}
fl2 = 0 
for i in \hlkwd{range}(1, \hlkwd{len}(vec)):
  fl2 += vec[i] 
fl2 = fl2 + vec[0] 
\hlkwd{print}(\hlkwd{Decimal}(fl2))

\hlcom{## 1 }
\hlcom{## 1.0000000000010000889005823410116136074066162109375}
\end{alltt}
\end{kframe}
\end{knitrout}

The first for loop returns a wrong value (1) like in R.

On the other hand, the second for loop returns the right value.


\section*{Number 2 (e)}

Apparently the results suggest that sum() does not simply sum the
numbers from left to right.

The result should have been the same in two for loops if it worked
that way.

The difference comes from the fact that we add 1 from the beginning
or at last.

Thus we can notice that sum() is more accurate when we sum up numbers
with same magnitudes.

It tries to be accurate up to 16 decimal places.


\section*{Number 2 (f)}


\end{document}
